\chapter{Number Theory}

\section{Induction}

Given some statement involving
a natural number $n$, induction
can be used to prove the statement
for all $n$.
\begin{description}
    \item[Base Case] The first step is to prove the
    statement for the first value of $n$, usually
    1 or 0.

    \item[Inductive Step] The second step proves
    the statement by first assuming the statement
    is true for $n$, then showing the statement
    is true for $n + 1$.
\end{description}

\subsection{Example}

Statement:
$$1 + 2 + 3 + ... + n = \frac{n(n + 1)}{2}$$

\noindent
Proof:
\begin{description}
    \item[Base Case]
    \begin{equation*}
        \begin{split}
        1 & = \frac{1(1 + 1)}{2} \\
        1 & = 1
        \end{split}
    \end{equation*}

    \item[Inductive Step] Assume true for $n$:
    $$1 + 2 + 3 + ... + n = \frac{n(n + 1)}{2}$$
    Now, prove for $n + 1$:
    \begin{equation}
        \begin{split}
        1 + 2 + 3 + ... + n + (n + 1) &= \frac{(n + 1)(n + 2)}{2} \\ 
        \overbrace{1 + 2 + 3 + ... + n}^{n(n + 1)/2} + n + 1 &=
        \frac{(n + 1)(n + 2)}{2} \\
        \frac{n(n + 1)}{2} + n + 1 &= \frac{(n + 1)(n + 2)}{2} \\
        \frac{n(n + 1) + 2(n + 1)}{2} &= \frac{(n + 1)(n + 2)}{2} \\
        \frac{(n + 1)(n + 2)}{2} &= \frac{(n + 1)(n + 2)}{2} \\
        \end{split}
    \end{equation*}

\end{description}

\section{Divisibility}

$a | b$ is read ''a divides b'', meaning that $a/b$
is an integer.

\begin{description}
    \item[Definition] $a | b$ iff there exists an integer $q$
    such that $aq = b$.

    \item[Theorems]
        \begin{enumerate}
            \item
            If $a | b$ and $b | c$, then $a | c$. Proof:
            $$aq_1 = b, bq_2 = c$$
            $$a\underbrace{q_1q_2}_{\text{an integer}} = c$$
            $$a | c$$
        \end{enumerate}

\end{description}
