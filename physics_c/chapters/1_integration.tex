\chapter{Integration \& Applications}

\section{Rules of Integration}

\subsection{Sum Rule}
The integral of a sum is the sum of the integrals:

$$\int (f(x) + g(x))dx = \int f(x)dx + \int g(x)dx$$

\subsection{Constant Multiple Rule}
You can ''pull out'' a constant from inside an integral:

$$\int kf(x)dx = k \int f(x)dx$$

\subsection{Power Rule}
For $n \ne -1$:
$$\int x^n dx = \frac{x^{n + 1}}{n + 1} + C$$

\begin{description}
    \item[Example 1.]
    $$\int x^4dx = \frac{x^5}{5} + C + C$$
    
    \item[Example 2.]
    $$\int \sqrt{x}dx = \frac{2x^{3/2}}{3} + C$$
\end{description}

\subsection{Application to motion}

To go back from acceleration to velocity, and velocity
to displacement, integrate.

$$v(t) = \int a(t)dt$$
$$d(t) = \int v(t)dt$$

We can derive our standard equations of motion for constant
acceleration with integration:

$$a(t) = a$$
$$v(t) = \int a(t)dt = at + C = v_0 + at$$
$$d(t) = \int v(t)dt = v_0t + \frac{1}{2}at^2 + C =
d_0 + v_0t + \frac{1}{2}at^2$$
